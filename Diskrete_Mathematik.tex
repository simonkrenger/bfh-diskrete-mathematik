%Simon Krenger: Diskrete Mathematik
\documentclass{report}
\usepackage{amsmath}
\usepackage{amsthm}
\usepackage{amssymb}
\usepackage[utf8]{inputenc} 
\usepackage{graphicx}

\newtheorem{mydef}{Definition}
\newtheorem{myexample}{Beispiel}
\newtheorem{myproof}{Beweis}

\title{Diskrete Mathematik}
\author{Simon Krenger}

\begin{document}
\maketitle
\chapter{Logik (Boolesche Algebra)}
Nach George Bool, 1815 bis 1864, Cork (Irland)
\section{Aussagen}
Wir betrachten Aussagen (Sätze), die entweder wahr (1) oder falsch (0) sind.

\begin{quote}Heute ist Freitag \(\to\) wahr\end{quote}
\begin{quote}Morgen schneit es in Bern \(\to\) falsch\end{quote}
\begin{quote}Schauen Sie einmal! \(\to\) keine Aussage\end{quote}
Aussagen bezeichnen wir mit a, b, c, d, …
\begin{mydef}
Ist a eine Aussage, somit heisst  \(\lnot\)a die \underline{Negation} von a
\end{mydef}
\begin{myexample}a: Xaver isst gerne Kuchen
\(\lnot\)a: Xaver isst nicht gerne Kuchen\end{myexample}

\section{Konjunktion}
Wir verbinden zwei Aussagen a, b mit Hilfe von “und” zu einer einzigen Aussage
\begin{equation}a \land b\end{equation}
\begin{myexample}Morgen ist Sonntag \underline{und} ich werde ausschlafen\end{myexample}

Die Wahrheitstabelle von a \(\land\) b sind abhängig von denjenigen von a als auch von b. Dies stellen wir in einer \underline{Wahrheitswerttabelle} dar. Wir finden sofort die Regeln
\begin{equation}a \land \lnot a = falsch\end{equation}


\begin{mydef}Eine Aussage, die immer falsch ist, heisst \underline{Kontradiktion}.\end{mydef}

\begin{equation}a \land 1 = a\end{equation}
\begin{equation}a \land 0 = 0\end{equation}
Weiter finden wir Gesetze
\begin{quote}Kommutativgesetz (Vertauschungsgesetz)\end{quote}
\begin{equation}a \land b = b \land a\end{equation}
\begin{myproof}Wir beweisen mit einer Wahrheitswerttabelle
\begin{center}\begin{tabular}{c c | c c}
a & b & a \(\land\) b & b \(\land\) a\\
\hline
0 & 0 & 0 & 0  \\
0 & 1 & 0 & 0  \\
1 & 0 & 0 & 0 \\
1 & 1 & 1 & 1 \\
\end{tabular}\end{center}\end{myproof}
\begin{quote}Assoziativgesetz (Verbindungsgesetz)\end{quote}
\begin{equation}a \land (b \land c) = (a \land b) \land c\end{equation}
\begin{myproof}Wir beweisen mit einer Wahrheitswerttabelle
\begin{center}\begin{tabular}{c c c | c c}
a & b & c & a \(\land\) (b \(\land\) c) & (a \(\land\) b) \(\land\) c\\
\hline
0 & 0 & 0 & 0 & 0  \\
0 & 0 & 1 & 0 & 0  \\
0 & 1 & 0 & 0 & 0 \\
0 & 1 & 1 & 0 & 0 \\
1 & 0 & 0 & 0 & 0 \\
1 & 0 & 1 & 0 & 0 \\
1 & 1 & 0 & 0 & 0 \\
1 & 1 & 1 & 1 & 1 \\
\end{tabular}\end{center}\end{myproof}
\begin{quote}Idempotenzgesetz\end{quote}
\begin{equation}a \land a = a\end{equation}
\section{Disjunktion}
Zwei Aussagen a, b werden mit der Disjunktion "oder" zu einer neuen Aussage verbunden. Dafür schreiben wir:\begin{equation}a \lor b\end{equation}
und definieren
\begin{center}\begin{tabular}{c c | c}
a & b & a \(\lor\) b\\
\hline
0 & 0 & 0  \\
0 & 1 & 1  \\
1 & 0 & 1 \\
1 & 1 & 1 \\
\end{tabular}\end{center}
Nicht verwechseln mit "entweder oder" (XOR)! Wir finden die Regeln
\begin{equation}a \lor 1 = 1\end{equation}
\begin{equation}a \lor 0 = a\end{equation}
\begin{equation}a \lor \lnot a = 1\end{equation}
\begin{mydef}Eine Aussage, die stets wahr ist, heisst \underline{Tautologie}.\end{mydef}
Es gelten die Gesetze
\begin{quote}Kommutativgesetz\end{quote}
\begin{equation}a \lor b = b \lor a\end{equation}
\begin{quote}Assoziativgesetz\end{quote}
\begin{equation}a \lor (b \lor c) = (a \lor b) \lor c\end{equation}
\begin{quote}Idempotenzgesetz\end{quote}
\begin{equation}a \lor a = a\end{equation}
In der Algebra in \(\mathbb{R}\) gilt
\begin{equation}a(b+c) = ab+ac\end{equation}
was in der Logik zu
\begin{equation} \label{eq:distributivgesetz}a \land (b \lor c) = (a \land b) \lor (a \land c)\end{equation}
\begin{equation}a \lor (b \land c) = (a \lor b) \land (a \lor c)\end{equation}
dem \underline{Distributivgesetz} (Verteilungsgesetz) führt. Der folgende Beweis zeigt, dass die Gleichung \ref{eq:distributivgesetz} gilt.
\begin{myproof}Wir beweisen mit einer Wahrheitswerttabelle
\begin{center}\begin{tabular}{c c c | c c c c c}
a & b & c & b \(\lor\) c & a \(\land\) (b \(\lor\) c) & a \(\land\) b & a \(\land\) c & (a \(\land\) b) \(\lor\) (a \(\land\) c)  \\
\hline
0 & 0 & 0 & 0 & 0 & 0 & 0 & 0  \\
0 & 0 & 1 & 1 & 0 & 0 & 0 & 0  \\
0 & 1 & 0 & 1 & 0 & 0 & 0 & 0 \\
0 & 1 & 1 & 1 & 0 & 0 & 0 & 0 \\
1 & 0 & 0 & 0 & 0 & 0 & 0 & 0 \\
1 & 0 & 1 & 1 & 1 & 0 & 1 & 1 \\
1 & 1 & 0 & 1 & 1 & 1 & 0 & 1 \\
1 & 1 & 1 & 1 & 1 & 1 & 1 & 1 \\
\end{tabular}\end{center}
Das zweite Distributivgesetz kann analog dazu bewiesen werden.\end{myproof}
In der Logik gibt es zu jedem Gesetz ein \underline{duales Gesetz}. Dies entsteht durch wechseln von \(\lor\) zu \(\land\) und umgekehrt. Weiter finden wir
\begin{quote}Absorbtionsgesetz\end{quote}
\begin{equation}a \land (a \lor b) = a\end{equation}
\begin{equation}a \lor (a \land b) = a\end{equation}
\begin{myproof}Wir beweisen mit einer Wahrheitswerttabelle
\begin{center}\begin{tabular}{c c | c c}
a & b & a \(\lor\) b & a \(\land\) (a \(\lor\) b) \\
\hline
0 & 0 & 0 & 0  \\
0 & 1 & 1 & 0  \\
1 & 0 & 1 & 1  \\
1 & 1 & 1 & 1 \\
\end{tabular}\end{center}\end{myproof}
\begin{quote}Gesetz von de Morgan\end{quote}
\begin{equation}\lnot (a \land b) = \lnot a \lor \lnot b\end{equation}
\begin{equation}\lnot (a \lor b) = \lnot a \land \lnot b\end{equation}
Wir verwenden die Gesetze, um die Aussagen zu vereinfachen.
\begin{myexample}Folgende Beispiele zeigen, wie sich Aussagen mittels den oben genannten Gesetzen vereinfachen lassen.
\begin{enumerate}
\item $[a \land (b \lor a)] \lor \lnot a$\\
$= a \lor \lnot a$\\
$= 1$
\item $[\lnot (a \land b) \lor \lnot b] \land a$\\
$= (\lnot a \lor \lnot b \lor \lnot b) \land a$\\
$= (\lnot a \lor \lnot b) \land a$\\
$= (\lnot a \land a) \lor (\lnot b \land a)$\\
$= 0 \lor (\lnot b \land a)$\\
$= \lnot b \land a$
\item $(a \land b) \lor \lnot b$\\
$= (a \lor \lnot b) \land (b \lor \lnot b)$\\
$= (a \lor \lnot b) \land 1$\\
$= (a \lor \lnot b)$
\item $b \land [(a \land b) \lor ( \lnot a \land b)]$\\
$= b \land [b \land (a \lor \lnot a)]$\\
$= b \land (b \land 1)$\\
$= b \land b = b$
\end{enumerate}
\end{myexample}
\section{Implikation}
Mathematische Lehrsätze haben die Form "Wenn ein Dreieck rechtwinklig ist mit Hypothenuse $c$ und Katheten $a$, $b$, dann ist $c^2=a^2+b^2$"\\
Sie bestehen also aus Voraussetzung(en):
\begin{quote}Das Dreieck ist rechtwinklig\end{quote}
und Behauptung
\begin{quote}Es ist $a^2+b^2=c^2$\end{quote}
und einem Beweis
\begin{quote}\begin{proof}[Beweis] Gemäss "Indischer Beweis":
\begin{eqnarray}c^2&=&4\frac{ab}{2}+(a-b)^2 \nonumber \\
c^2&=&2ab+a^2-2ab+b^2\nonumber \\
c^2&=&a^2+b^2\end{eqnarray}\end{proof}\end{quote}
Im obrigen Beispiel haben wir einen direkten Beweis geführt. Von der Voraussetzung durch Rechnung zur Behauptung.\\
Wenn wir zwei Aussagen a, b mit "wenn a, dann b" oder "wenn a so b" oder "aus a folgt b (a impliziert b)" verknüpfen, so schreiben wir dafür
\begin{equation}a \to b\end{equation}
und definieren
\begin{center}\begin{tabular}{c c | c}
a & b & a \(\to\) b\\
\hline
0 & 0 & 1  \\
0 & 1 & 1  \\
1 & 0 & 0 \\
1 & 1 & 1 \\
\end{tabular}\end{center}
Wir finden sofort, das "aus a folgt b"
\begin{equation}a \to b = \lnot a \lor b\end{equation}
\begin{myexample}Vereinfache
\begin{enumerate}
\item $(a \to b) \to b$\\
$= (\lnot a \lor b) \to b = \lnot(\lnot a \lor b) \lor b$\\
$= (a \land \lnot b) \lor b = (a \lor b) \land (\lnot b \lor b)$\\
$= (a \lor b) \land 1 = (a \lor b)$
\item $b \to (a \to b)$\\
$= b \to (\lnot a \lor b) = \lnot b \lor (\lnot a \lor b)$\\
$= \lnot b \lor b \lor \lnot a = 1 \lor \lnot a = 1$
\item $[(a \lor c) \land (c \to a)] \lor (a \land \lnot b) \lor (a \land c) \lor [\lnot a \land (b \to c)]$\\
$=[(a \lor c) \land (\lnot c \lor a)] \lor (a \land \lnot b) \lor (a \land c) \lor [\lnot a \land (\lnot b \lor c)]$\\
$=[(a \lor c) \land (\lnot c \lor a)] \lor [a \land (\lnot v \lor c)] \lor [\lnot a \land (\lnot b \lor c)]$\\
$=[a \lor (c \land \lnot c)] \lor [(\lnot b \lor c) \land (a \lor \lnot a)]$\\
$=[a \lor 0] \lor [(\lnot b \lor c) \land 1]$\\
$=a \lor (\lnot v \lor c) = a \lor \lnot b \lor c$\\
$(=a \lor (b \to c))$
\end{enumerate}
\end{myexample}
Ein mathematischer Satz besteht aus Voraussetzung $a$, Behauptung $b$ und Beweis. Der Satz wird als $a \to b$ formuliert.\\
Der direkte Beweis ist eine Folge von Implikationen
\begin{equation}a \to x_1 \to x_2 \to x_3 \to ... \to b\end{equation}
\begin{myexample}Vereinfache
\begin{enumerate}
\item Voraussetzung: Ein Dreieck ABC mit Innenwinkel $\alpha$, $\beta$, $\gamma$\\
Behauptung: Die Innenwinkelsumme ist $180^{\circ}$, d.h. 
\begin{equation}\alpha+\beta+\gamma=180^{\circ}\end{equation}
\begin{proof}[Beweis]Wir beweisen mit einer Zeichnung:\begin{center}
\includegraphics[scale=0.5]{img/innenwinkel.png}
\end{center}
Wähle $p \parallel c$ durch C. Dann ist $\epsilon+\delta+\gamma=180^{\circ}$. Es ist $\alpha_1$= $\alpha_2$: Stufenwinkel an Parallelen und $\alpha_1$= $\alpha_3$: Wechselwinkel an Parallelen, eine weitere Voraussetzung.\\
Also ist $\alpha = \epsilon$ und $\beta = \delta$
und somit
\begin{equation}\alpha+\beta+\gamma=180^{\circ}\end{equation}
\end{proof}
\item Voraussetzung: Es ist mit $n \in \mathbb{N}, a \in \mathbb{R}$
\begin{center}$a^n := a \cdot a \cdot ... \cdot a$ (n Faktoren)\end{center}
die Potenz definiert.
Behauptung: \begin{equation}a^m \cdot a^n = a^{m+n}\end{equation}
\begin{proof}[Beweis]Wir zeigen auf, dass $m$ Faktoren mit $n$ Faktoren multipliziert werden. Durch die grundlegenden Rechengesetze können wir die Klammern wegfallen lassen
\begin{center}$a^m \cdot a^n$\\
$= (a \cdot a \cdot a \cdot ... \cdot a)(a \cdot a \cdot ... \cdot a)$\\
$= a \cdot a \cdot a \cdot a \cdot ... \cdot a$ (m+n Faktoren)\end{center}
\begin{equation}= a^{m+n} = a^m \cdot a^n\end{equation}
\end{proof}
\item \label{item:gegenbeispiel} Voraussetzung: $x, y \in \mathbb{R}$\\
Behauptung: \begin{equation} x^y = y^x \to x = y\end{equation} 
Die Behauptung ist falsch. Wollen wir zeigen, dass ein Satz falsch ist, so genügt ein einziges Beispiel, dass wir \underline{Gegenbeispiel} nennen, um die Behauptung zu widerlegen.\\
Gegenbeispiel: Für \ref{item:gegenbeispiel} ist das Gegenbeispiel $x=2, y=4$, denn $2^4 = 4^2 = 16$, aber $x \neq y$.
\end{enumerate}
\end{myexample}

\subsection{Umkehrung, Kontraposition}
\begin{mydef}Hat eine Aussage die Form
\begin{equation}a \to b\end{equation}
so heisst
\begin{equation}b \to a\end{equation}
die \underline{Umkehrung}.
\end{mydef}
Ist eine Aussage, ein Satz wahr, so muss die Umkehrung nicht wahr sein, wie zum Beispiel:
\begin{quote}"Wenn ich Geburtstag habe, so esse ich einen Kuchen"\end{quote}
\begin{quote}"Wenn ein Mensch glücklich ist, so trinkt er Sinalco"\end{quote}
Wir finden aber, dass
\begin{eqnarray}\lnot b \to \lnot a = \lnot(\lnot b) \lor \lnot a \nonumber\\
= b \lor \lnot a = \lnot a \lor b = a \to b\end{eqnarray}
\begin{mydef}Wir nennen
\begin{equation}\lnot b \to \lnot a\end{equation}
die \underline{Kontraposition} von
\begin{equation}a \to b\end{equation}
\end{mydef}
Wir haben gezeigt, dass $\lnot b \to \lnot a = a \to b$ ist, was bedeutet, dass bei einem wahren Satz auch dessen Kontraposition wahr ist.
\begin{quote}{Satz}: "Wenn es heute Freitag ist, so gehe ich ein Bier trinken."\end{quote}
\begin{quote}{Kontraposition}: "Wenn ich nicht ein Bier trinken gehe, so ist heute Freitag"\end{quote}
Manchmal ist der direkte Beweis eines Satzes zu schwierig oder nicht möglich, dann beweisen wir die Kontraposition.
\begin{quote}{Satz}: Ist $n \in \mathbb{N}$ und $n^2$ eine gerade Zahl, so ist $n$ auch eine gerade Zahl.\end{quote}
\begin{myproof}Der direkte Beweis
\begin{eqnarray}n^2 &=& 2p \land p \in \mathbb{N} \nonumber \\
\to n &=& \sqrt{2} \cdot \sqrt{p}\end{eqnarray}
gelingt nicht. Grund dafür ist, dass eine irrationale Zahl ($\sqrt{2}$) per Definition ein nichtperiodischer, nichtendlicher Dezimalbruch ist.\end{myproof}
Also beweisen wir die Kontraposition:
\begin{quote}{Kontraposition}: "Ist $n \in \mathbb{N}$ und $n$ ungerade, so ist auch $n^2$ ungerade"\end{quote}
\begin{myproof}\begin{eqnarray} 
n &=& 2p + 1 \quad \land p \in \mathbb{N}_0 \nonumber \\
\to n^2 &=& (2p + 1)^2 \nonumber \\
n^2 &=& 4p^2 + 4p + 1 \nonumber \\
n^2 &=& 2 (2p^2 + 2p) + 1\label{eq:nungerade} \end{eqnarray}
Also ist $n^2$ eine ungerade Zahl.\qed\end{myproof}

\section{Aequivalenz}
Wenn zwei Aussagen gleichwertig (aequivalent) sind, wenn also
\begin{equation}(a \to b) \land (b \to a)\end{equation}
so schreiben wir dafür
\begin{equation}a \iff b\end{equation}
und finden die Wahrheitswerte
\begin{center}\begin{tabular}{c c | c}
a & b & a \(\iff\) b\\
\hline
0 & 0 & 1 \\
0 & 1 & 0 \\
1 & 0 & 0 \\
1 & 1 & 1 \\
\end{tabular}\end{center}
Wir finden die Umformung
\begin{eqnarray}
a \iff b &=& (a \to b) \land (b \to a)\nonumber \\
&=& (\lnot a \lor b) \land (\lnot b \lor a)\nonumber \\
&=& (\lnot a \land b) \lor (\lnot a \land a) \lor (b \land \lnot b) \lor (a \land b) \nonumber \\
&=& (a \land b) \lor (\lnot a \land \lnot b)
\end{eqnarray}
Ausserdem ist 
\begin{equation}a \iff b = \lnot (a \veebar b)\end{equation}
also
\begin{eqnarray}a \veebar b &=& [(a \land b) \lor (\lnot a \land \lnot b)] \nonumber \\
&=&\lnot(a \land b) \land \lnot (\lnot a \land \lnot b) \nonumber \\
&=&(\lnot a \lor \lnot b) \land (a \lor b)\end{eqnarray}
\begin{myexample}Vereinfache
\begin{enumerate}
\item $(\lnot a \lor \lnot b) \land (a \lor b)$\\
$=a \veebar b$ nach obriger Herleitung
\item $(a \land \lnot b \land \lnot c) \lor (a \land b \land c)$\\
$= a \land [(\lnot b \land \lnot c) \lor (b \land c)]$\\
$= a \land (b \iff c)$
\end{enumerate}
\end{myexample}
Wenn wir in der Mathematik einen Satz finden, dessen Umkehrung auch wahr ist, so wählen wir die Formulierung mit
\begin{quote}"dann und nur dann" oder "genau dann"\end{quote}
im Englischen
\begin{quote}"if and only if" oder "iff"\end{quote}
\begin{myexample}Folgende Beispiele zeigen solche Sätze
\begin{enumerate}
\item Zwei Dreiecke $ABC$ und $A_1B_1C_1$ sind \underline{genau dann} ähnlich, \underline{wenn} zwei Winkel gleich sind. TODO

\item Sind $a$, $b$ reelle Zahlen, so ist das Produkt \underline{dann und nur dann} 0, \underline{wenn} $a$ oder $b$ Null ist.
\begin{quote}{Voraussetzung}: $a, b \in \mathbb{R}$\end{quote}
\begin{quote}{Satz}: $(a \cdot b = 0) \iff (a = 0 \lor b = 0)$\end{quote}
\begin{quote}{Anwendung}:\begin{eqnarray}a^2 - 7x + 12 &=& 0\quad(x \in \mathbb{R}) \nonumber \\
(x-3)(x-4)&=&0 \nonumber \\
\to x-3=0 \quad &\lor& \quad x-4=0 \nonumber \\
x_1 = 3 &\quad& x_2 = 4\end{eqnarray}\end{quote}
\end{enumerate}
\end{myexample}
Wie zeigen wir, dass zwei Terme gleich sind?
\begin{quote}{Behauptung}: \begin{equation}\sin{2\alpha} = 2\sin{\alpha}\cos{\alpha}\end{equation} \end{quote}
Wir wählen die linke Seite und formen diese so lange um, bis die rechte Seite entsteht (oder umgekehrt).\\Es ist \underline{falsch}
\begin{equation}\sin{2\alpha} = 2\sin{\alpha}\cos{\alpha}\end{equation}
so lange umzuformen, bis eine Identität wie z.B. $1 = 1$ entsteht!

\begin{myexample}Richtig ist
\begin{eqnarray}\sin{2\alpha} &=& \sin{\alpha + \alpha} \nonumber \\
\mbox{denn}\quad \sin{\alpha + \beta} &=& \sin{\alpha}\cos{\beta}+\sin{\beta}\cos{\alpha}\nonumber \\
\sin{\alpha + \alpha} &=& \sin{\alpha}\cos{\alpha}+\sin{\alpha}\cos{\alpha} \nonumber \\
&=& 2\sin{\alpha}\cos{\alpha}\end{eqnarray} \qed\end{myexample}
Manchmal gelingt es nicht, die linke Seite in die rechte Seite umzuformen. Dann verwenden wir die Eigenschaft
\begin{quote}"Wenn $l=x$ und $r=x$, so ist $l=r$\end{quote}
Wir formen also die linke Seite zuerst einmal um und dann \underline{unabhängig davon} die rechte Seite und hoffen, dass wir beide Male das gleiche Resultat ($x$) erhalten.
\begin{myexample}WIr versuchen, dieses Konzept anzuwenden:\\\\
Voraussetzung: \begin{equation}\tan{\delta} = \frac{\sin{\delta}}{\cos{\delta}} \quad\mbox{und}\quad\cot{\delta} = \frac{1}{\tan{\delta}}\end{equation}
Behauptung: \begin{equation}\tan{\delta} + \cot{\delta} = \frac{2}{\sin{2\delta}}\end{equation}
Beweis: \begin{enumerate}
\item \begin{eqnarray}& &\tan{\delta} + \frac{1}{\tan{\delta}}
=\frac{\sin{\delta}}{\cos{\delta}} + \frac{\cos{\delta}}{\sin{\delta}} \nonumber \\
&=&\frac{\sin^2{\delta} + \cos^2{\delta}}{\sin{\delta} \cdot \cos{\delta}} 
=\frac{1}{\sin{\delta} \cdot \cos{\delta}}\end{eqnarray}
\item \begin{eqnarray}& &\frac{2}{\sin{2\delta}} = \frac{2}{2\sin{\delta} \cdot \cos{\delta}}
=\frac{1}{\sin{\delta} \cdot \cos{\delta}}\end{eqnarray}\qed
\end{enumerate}\end{myexample}
Genau gleich behandeln wir Behauptungen der Logik wenn es um die Aequivalenz zweier Aussagen geht.
\begin{myexample}\begin{enumerate}
\item Behauptung: \begin{equation}[\lnot(a \lor b) \land a] \iff [\lnot (a \lor b) \land b]\end{equation}
Beweis:\begin{enumerate}
\item \begin{eqnarray}\lnot (a \lor b) \land a &=& \lnot a \land \lnot b \land a \nonumber \\
= \lnot a \land a \land \lnot b &=& 0 \land \lnot b = 0\end{eqnarray}
\item \begin{eqnarray}\lnot (a \lor b) \land b &=& \lnot a \land \lnot b \land b \nonumber \\
= \lnot a \land b \land \lnot b &=& \lnot a \land b = 0\end{eqnarray}\end{enumerate}Beide Terme sind aequivalent\qed
\item Behauptung: \begin{equation}a \to (b \land c) = (a \to b) \land (a \to c)\end{equation}
Beweis: \begin{eqnarray}\lnot a \lor (b \land c) &=& (\lnot a \lor b) \land (\lnot a \lor c) \nonumber \\
&=& (a \to b) \land (a \to c)\end{eqnarray}\end{enumerate}\end{myexample}

\section{Logische Schlüsse}
Wir gehen aus von verschiedenen \underline{Prämissen} wie
\begin{eqnarray}\mbox{Prämisse 1} & & p_1 = a \land b \nonumber \\
\mbox{Prämisse 2}& & p_2 = \lnot a \nonumber \\
\mbox{Prämisse 3}& & p_3 = a \land \lnot b\end{eqnarray}
und ziehen daraus eine \underline{Konklusion} $k : a \lor b$. Nun fragen wir uns, ob die Konklusion bei diesen Prämissen richtig ist. Ist dies der Fall, so sprechen wir von einem logischen Schluss (wenn also das die richtige Konklusion ist).
\\\\Es muss also
\begin{equation}(p_1 \land p_2 \land ... \land p_n) \to k = 1\end{equation}
eine Tautologie sein. Im Beispiel ist also
\begin{equation}[(a \land b) \land \lnot a \land (a \land \lnot b)] \to (a \lor b)\end{equation}
so lange umgeformt werden, bis erkenntlich ist, ob eine Tautologie vorliegt oder nicht.
\begin{eqnarray}& &[(a \land b) \land \lnot a \land (a \land \lnot b)] \to (a \lor b) \nonumber \\
&=&(a \land b \land \lnot a \land a \land \lnot b) \to (a \lor b) \nonumber \\
&=&0 \to (a \lor b) \nonumber \\
&=&\lnot 0 \lor (a \lor b) = 1 \lor (a \lor b) = 1\end{eqnarray}
und damit liegt ein logischer Schluss vor.\\\\
In der Logik schreiben wir Prämissen und Konklusion untereinander wie zum Beispiel
\begin{equation}\frac{a \to b \quad\quad a \land b \to c \quad\quad c}{a}\end{equation}
\begin{myexample}Handelt es sich hierbei um einen logischen Schluss?\begin{eqnarray}& &[(a \to b) \land \{(a \land b) \to c\} \land c] \to a \nonumber \\
&=&[(\lnot a \lor b) \land \{\lnot (a \land b) \lor c\} \land c] \to a \nonumber \\
&=&\lnot [(\lnot a \lor b) \land (\lnot a \lor \lnot b \lor c) \land c] \lor a \nonumber \\
&=&\lnot [(\lnot a \lor b) \land c] \lor a \nonumber \\
&=&\lnot [(\lnot a \land c) \lor (b \land c)] \lor a\nonumber \\
&=&[\lnot (\lnot a \land c) \land \lnot (b \land c)] \lor a \nonumber \\
&=&[(a \lor \lnot c) \land (\lnot b \lor \lnot c)] \lor a \nonumber \\
&=&(a \lor \lnot c \lor a) \land (a \lor \lnot b \lor \lnot c) \nonumber \\
&=&(\lnot c \lor a) \land (\lnot b \lor \lnot c \lor a) \nonumber \\
&=&(\lnot c \lor a) \land \lnot b\end{eqnarray}
Also kein logischer Schluss \qed\end{myexample}
Verschiedene bekannte logische Schlüsse besitzen einen Namen, wie zum Beispiel die Folgdenden:
\begin{enumerate}
\item \underline{modus ponens} (Abtrennungsregel)
\begin{equation}\frac{a \to b \quad\quad a}{b}\end{equation}
ist ein logischer Schluss, denn
\begin{eqnarray}& &[(a \to b) \land a] \to b \nonumber \\
&=&[(\lnot a \lor b) \land a] \to b \nonumber \\
&=&(a \land b) \to b \nonumber \\
&=&\lnot (a \land b) \lor b \nonumber \\
&=&\lnot a \lor \lnot b \lor b = \lnot a \lor 1 = 1\end{eqnarray}
Es ist die Art und Weise, wie wir einen mathematischen Satz $a \to b$ anwenden.
\begin{myexample}Beispielsweise Kosinussatz:\\
$a \to b$: In einem Dreieck $ABC$ gilt $c^2 = a^2 + b^2 - 2ab \cdot \cos{\gamma}$ \\
$a$: $a=10, b=7, \gamma = 70$\\
dann tritt $b$ ein, d.h. c kann nun berechnet werden.\end{myexample}
\item \underline{modus tollens} (Aufhebende Schlussweise)
\begin{equation}\frac{a \to b \quad\quad \lnot b}{\lnot a}\end{equation}
ist ein logischer Schluss, denn
\begin{eqnarray}& &[(a \to b) \land \lnot b] \to \lnot a \nonumber \\
&=&[(\lnot a \lor b) \land \lnot b] \to \lnot a \nonumber \\
&=&[(b \land \lnot b) \lor (\lnot a \land \lnot b)] \to \lnot a \nonumber \\
&=&(\lnot a \land \lnot b) \to \lnot a \nonumber \\
&=&\lnot (\lnot a \land \lnot b) \lor \lnot a \nonumber \\
&=&a \lor b \lor \lnot a = 1 \lor b = 1\end{eqnarray}
\item \underline{reductio ad absurdum} (zurückführen auf einen WIderspruch)
\begin{equation} \frac{a \to (b \land \lnot b)}{\lnot a}\end{equation}
ist ein logischer Schluss, denn
\begin{eqnarray}& &[a \to (b \land \lnot b)] \to \lnot a \nonumber \\
&=&[a \to 0] \to \lnot a \nonumber \\
&=&[\lnot a \lor 0] \to \lnot a \nonumber \\
&=&\lnot a \to \lnot a = a \lor \lnot a = 1\end{eqnarray}
Dieser logische Schluss führt uns zum \underline{Beweis mit Gegenannahme}.\\\\
Wollen wir beweisen, dass ein Satz $s$ wahr ist und gelingt uns dies nicht mit einem direkten Beweis oder mit einem Beweis mit Kontraposition, so wählen wir die Gegenannahme:
\begin{quote}$\lnot s$ ist wahr\end{quote}
und zeigen, dass dies zu einem Widerspruch führt wie $\lnot b \land b$ oder $1 = 2$ oder ähnlich.\\\\
Dann sagt uns die "reductio ad absurdum", dass meine Gegenannahme falsch ist und damit die Aussage $s$ wahr ist.\end{enumerate}
\begin{myexample}\begin{enumerate}
\item Satz: "Es gibt unendlich viele Primzahlen"\\
Beweis mit Gegenannahme (Euklid, ca. 300 v.Chr., Alexandria):
\begin{quote}"Es gibt nur endlich viele Primzahlen"\end{quote}
\begin{equation}p_1 < p_2 < p_3 < ... < p_{n-1} < p_n\end{equation}
wobei $p_n$ die Grösste sei.\\
Nun bilden wir eine neue Zahl
\begin{equation}z = p_1 \cdot p_2 \cdot p_3 \cdot ... \cdot p_n + 1\end{equation}
die sicher keine der Zahlen $p_1, p_2, p_3, ..., p_n$ als Primfaktoren besitzt.\\
Nun ist $z$ entweder
\begin{enumerate}\item eine Primzahl, dann ist dies ein Widerspruch
\item keine Primzahl und damit in Primfaktoren zerlegbar. Es muss also neben $p_1, p_2, ..., p_n$ einen weiteren Primfaktor geben, dies ist ein Widerspruch\end{enumerate}
zur Gegenannahme.\\
Also ist die Gegenannahme falsch und damit die ursprüngliche Behauptung wahr. \qed

\item Behauptung: $\sqrt{2}$ ist irrational\\
Beweis mit Gegenannahme:
\begin{quote}$\sqrt{2}$ ist rational\end{quote}
also ist $\sqrt{2} = \frac{p}{q} \land p, q \in \mathbb{N}$ und vollständig gekürzt. Somit
\begin{eqnarray}2&=&\frac{p^2}{q^2} \nonumber \\
p^2&=&2q^2 \label{eq:sqrtirrational} \end{eqnarray}
Also ist $p^2$ eine gerade Zahl und damit auch $p$ (Beweis siehe \ref{eq:nungerade}). Somit ist $p = 2x \land p \in \mathbb{N}$, was eingesetzt in \ref{eq:sqrtirrational} zu
\begin{equation}(2x)^2 = 2q^2\end{equation}
führt. Weiter ist
\begin{eqnarray}4x^2&=&2q^2 \nonumber \\
2x^2&=&q^2\end{eqnarray}
Also ist $q^2$ gerade und damit auch q. Somit ist
\begin{equation}q = 2y \land y \in \mathbb{N}\end{equation}
WIr haben also gefunden
\begin{equation}\sqrt{2} = \frac{p}{q} = \frac{2x}{2y}\end{equation}
und damit erhalten wir einen Widerspruch zu "vollständig gekürzt". Somit ist die Gegenannahme falsch und damit die Behauptung richtig.
\end{enumerate}\end{myexample}
Einen Beweis mit Gegenannahme nennen wir auch einen \underline{indirekten Beweis}. Dieses Beweisverfahren können wir auch für logische Schlüsse anwenden.\\\\
Ist
\begin{equation}\frac{a \land \lnot b \quad a \to b}{a \lor b}\end{equation}
ein logischer Schluss?
\begin{quote}Gegenannahme: Es ist liegt kein logischer Schluss vor und damit ist
\begin{equation}[(a \land \lnot b) \land (a \to b)] \to (a \lor b) = 0\end{equation}\end{quote}
Nun zeigen wir, dass die Gegenannahme zu einem Widerspruch führt. Wir haben die Aussage
\begin{equation}x \to y = 0 \nonumber\end{equation}
Also muss $x=1$ und $y=0$ sein.\\\\
Es ist $x = p_1 \land p_2 \land ... \land p_n$ (Alle Prämissen und damit muss auch
\begin{equation}p_1 = p_2 = ... = p_n = 1 \nonumber \end{equation}
sein. Um den Widerspruch zu sehen, machen wir eine Tabelle:
\begin{equation}\begin{tabular}{c | c | c | c | c |}
& $a \land \lnot b$ & $a \to b$ & $\to$ & $a \lor b$ \\ \hline
1) & \underline{1} & 1 & & 0 \\ \hline
2) & & & & $a=0, b=0$ \\ \hline
3) &\underline{$0 \land 1 = 0$} & & & \\ \hline
\end{tabular}\end{equation}Bei den unterstrichenen Werten haben wir einen Widerspruch hergeführt. Die Gegenannahme ist falsch, also liegt ein logischer Schluss vor.
\begin{myexample}
\begin{enumerate}
\item Ist
\begin{equation}\frac{a \land \lnot d \quad \lnot a \lor c \quad (b \land \lnot c) \to a}{a \lor c \lor d}\end{equation}ein logischer Schluss?\\\\
\underline{Gegenannahme}:
\begin{equation}\{(a \land \lnot d) \land (\lnot a \lor c) \land [(b \land \lnot c) \to a]\} \to (b \lor c \lor d) = 0\end{equation}also
\begin{equation}\begin{tabular}{c | c | c | c | c | c |}
& $a \land \lnot d$ & $\lnot a \lor c$ & $(b \land \lnot c) \to a$ & $\to$ & $b \lor c \lor d$ \\ \hline
1) & 1 & \underline{1} & 1 & & 0 \\ \hline
2) & & & & & $b=0, c=0, d=0$ \\ \hline
3) & $a=1$ & & & & \\ \hline
4) & & \underline{$0 \lor 0 = 0$} & & & \\ \hline
\end{tabular}\end{equation}Bei den unterstrichenen Werten haben wir einen Widerspruch hergeführt. Die Gegenannahme ist falsch, also liegt ein logischer Schluss vor.
\item Wir untersuchen, ob
\begin{equation}\frac{a \to \lnot b \quad \lnot c \to d \quad c \to a \quad e \to b}{b \to (d \lor c)}\end{equation}ein logischer Schluss ist. Gegenannahme:
\begin{equation}[(a \to \lnot b) \land (\lnot c \to d) \land (c \to a) \land (e \to b)] \to [b \to (d \lor e)] = 0\end{equation}
also
\begin{equation}\begin{tabular}{c | c | c | c | c | c | p{1.8cm}|}
& $a \to \lnot b$ & $\lnot c \to d$ & $c \to a$ & $e \to b$ & $\to$ & $b \to (d \lor e)$ \\ \hline
1) & 1 & \underline{1} & 1 & 1 & & 0 \\ \hline
2) & & & & & & \parbox{1.8cm}{$\\1 \to 0,\\b=1,\\d \lor e=0,\\ d=e=0$} \\ \hline
3) & $a=0$ & & $c=0$ & & & \\ \hline
4) & & \underline{$1 \to 0 \neq 1$} & & & & \\ \hline
\end{tabular}\end{equation}Also liegt ein logischer Schluss vor.\end{enumerate}
\end{myexample}
\section{Prädikatenlogik}
Einige Aussagen wie
\begin{itemize}
\item Informatiker(innen) besitzen einen Laptop
\item Katzen schnurren
\item Hunde bellen
\item $a \cdot b = b \cdot a$\end{itemize}
verlangen eine Präzisierung wie
\begin{itemize}
\item Nicht alle Informatiker(innen) besitzen einen Laptop
\item Einige Katzen schnurren
\item Alle Hunde bellen
\item Für alle $a,b \in \mathbb{R}$ ist $a \cdot b = b \cdot a$\end{itemize}
WIr brauchen also ein \underline{Prädikat} (Aussage) über Grössen aus einer bestimmten \underline{Menge} und einen \underline{Quantor}.
Wir nennen $\forall$ den \underline{Allquantor}. Damit bedeutet
\begin{quote}$x \in M: \forall x (P(x))$\\
(Für alle $x$ gilt $P(x)$")\end{quote}
dass alle Elemente der Menge $M$ das Prädikat $P$ besitzen.
\begin{myexample}Prädikatenlogik
\begin{quote}
$M = \{x \quad|\quad x \quad \mbox{ist ein Hund}\}$ \\
$B(x) : \mbox{x bellt}$\end{quote}
und es ist $x \in M : \forall x (B(x))$\end{myexample}
\end{document}
